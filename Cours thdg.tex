\documentclass{report}
\usepackage[utf8]{inputenc}
\usepackage[french]{babel}
\usepackage{parskip}
\usepackage[a4paper,width=160mm,top=25mm,bottom=25mm]{geometry}
\usepackage{graphicx}
\usepackage{amsfonts} %police
\usepackage{setspace}
\usepackage{stmaryrd} % pour les doubles crochet des ensembles d'entier
\usepackage[leftbars]{changebar}
\usepackage{mathtools} % pour les symboles de maths un peu tricky
\usepackage{tcolorbox} % pour le joli environnement
\usepackage{xcolor} % pour des encadrement en couleurs \fcolorbox{side}{bg}{contenu}
\usepackage{tikz}% pour les diagrammes
\usepackage{tikz-cd}
\usetikzlibrary{calc,decorations}
\usepackage{enumitem} % pour modifier facilement les items/ ennumérations
\usepackage[T1]{fontenc}
\usepackage[pdftex]{hyperref} %pour les liens ds le poly
\usepackage{framed}%pour faire les bares de preuves
\usepackage{amsfonts, amsmath, amssymb, ragged2e, mathtools, fancybox, moresize, mathrsfs, geometry, upgreek, stmaryrd, fancyhdr, lastpage, extarrows, halloweenmath, calrsfs, mathabx, mathbbol}

\renewenvironment{leftbar}{%
  \def\FrameCommand{\vrule width 0.4pt \hspace{10pt}}%
  \MakeFramed {\advance\hsize-\width \FrameRestore}}%
 {\endMakeFramed}%une commande obsucure du net


\newenvironment{definition}[1][]{
    \begin{tcolorbox}[colframe= white]
    \textbf{Définition :} 
    #1 \par
    }
    {\end{tcolorbox}}

\newenvironment{preuve}{\vspace*{0.5cm}
    \begin{leftbar}
    \noindent\textbf{Preuve :}\par}{
    \begin{flushright}
    $\square$
    \end{flushright}
    \end{leftbar}
}

%\newenvironment{preuve}{\begin{tcolorbox}[opacityfill=0,colframe= white]
%    \textbf{Preuve :}
%\par }
%    {\begin{flushright}
%    $\square$
%    \end{flushright}
%    \end{tcolorbox}}

\newenvironment{prop}{\begin{tcolorbox}[colframe= white]
    \textbf{Propriété :}
     \par}
    {\end{tcolorbox}}

\newenvironment{exemple}{\begin{tcolorbox}[colback=gray!10,colframe= white]
    \textbf{Exemple :}
     \par}
    {\end{tcolorbox}}

\newenvironment{exo}{\begin{tcolorbox}[colframe= white]
    \textbf{Exercice :}
    \par}
    {\end{tcolorbox}}

\newenvironment{corollaire}{\begin{tcolorbox}[colframe= white]
    \textbf{Corollaire :} \par}
    {\end{tcolorbox}}

\newenvironment{lemme}[2][white]{\begin{tcolorbox}[colframe= #1]
    \textbf{Lemme :} #2  \par}
    {\end{tcolorbox}}

\newenvironment{theoreme}[1][]{
    \begin{tcolorbox}[]
    \textbf{Théorème :} #1  \par} 
    {\end{tcolorbox}}

\newcommand{\remarque}{
    \noindent\textbf{Remarque :} \par
}

\newcommand{\mat}[2]{
    \mathcal{M}_{#1}(\mathbb{#2})
}

\newcommand{\gl}[2]{
    \mathcal{\mathbb{G}\mathbb{L}}_{#1}(\mathbb{#2})
}

\newcommand{\im}[0]{\textrm{Im}}

\newcommand{\fonction}[5]{
    \begin{array}{l|rcl}
    #1: & #2 & \longrightarrow & #3 \\
        & #4 & \longmapsto & #5 
    \end{array}
}

\newcommand{\fctsarg}[3]{
    \begin{array}{lrcl}
    #1: & #2 & \longrightarrow & #3 \\
    \end{array}
}
\newcommand{\ssi}{\Longleftrightarrow}
\newcommand{\N}{\mathbb{N}}
\newcommand{\Z}{\mathbb{Z}}
\newcommand{\R}{\mathbb{R}}
\newcommand{\Q}{\mathbb{Q}}
\newcommand{\C}{\mathbb{C}}
\newcommand{\znz}{\Z/n\Z}
\newcommand{\dn}{D_{n}}
\newcommand{\sn}{\mathcal{S}_{n}}
\newcommand{\an}{\mathcal{A}_{n}}
\newcommand{\unn}{\{1, ..., n\}}
\newcommand{\sign}{\epsilon(\sigma)}
\newcommand{\dsp}{\displaystyle}

\newcommand{\mc}{\mathcal}
\newcommand{\D}{\right}
\newcommand{\G}{\left}
\newcommand{\jus}{\ju\vspace{0.5cm}}
\newcommand{\ju}{\justify}

\newcommand{\fonctions}[5]{\begin{displaymath}#1:\left| \begin{array}{ccc}
 #2 & \longrightarrow & #3 \\
    #4 & \longmapsto & #5 \end{array}\right.\end{displaymath}}

\title{Théorie des groupes}
\author{Bcp de monde \dots }
\date{September 2024}

\begin{document}

\maketitle

\tableofcontents

\chapter*{Exemples en tout genres}



\section{Section1}
\begin{definition}{distance}
    mdr a definition d'une distance
\end{definition}
\subsection{Sous-section1}


\begin{preuve}
    exemple de preuve
\end{preuve}

\subsection{Comment faire un lemme}
\begin{lemme}[black]{}
avec la box noire sans nom
\end{lemme}

\begin{lemme}{nom}
    sasn la box mais avec le nom
\end{lemme}



\chapter{Notion de groupe, morphisme, produit direct}


\section{Groupes, sous-groupes, exemples}

\subsection{Définitions}
\begin{definition}{Groupe}
Un groupe est un ensemble non vide G munis d'une loi $\ast$ telle que :
\begin{enumerate}[label=(\roman*)]
\item $\ast$ est associative
\item $\ast$ possède un neutre $e\in G$
\item Tout élément possède un inverse pour $\ast$
\end{enumerate}

\end{definition}

\begin{definition}[Groupe abélien]
Un groupe G est dit \underline{abélien} si : $\forall (x,y)\in G,~xy=yx$
\end{definition}

\subsection{Sous-groupes}

\begin{definition}[Sous-groupe]
Un sous-ensemble H de G est appelé sous-groupe si :
\begin{itemize}[label=$\bullet$]
\item $e\in H$
\item $\forall x,y \in H, xy^{1}\in H$
\end{itemize}
\end{definition}

\begin{definition}[Groupe fini]
G est dit fini si il est cardinal fini, on note alors $o(G) = |G|$, appelé ordre de G.
\end{definition}


\subsection{Sous-groupe engendré}

\begin{definition}[Sous-groupe engendré par une partie]
Soient G un groupe et $S\subset G$\\
Soit $G_{S}$ l'ensemble des sous groupes de G qui contiennent S. \\
{\color{white}-}\\
On appelle sous groupe engendré par S l'ensemble : $ \dsp\langle S\rangle := \bigcap _{H\in G_{S}}H$\\
{\color{white}-}\\
Si de plus $\langle S\rangle = G$ on dit que S est une partie génératrice de G ou que S engendre G
\end{definition}

\begin{definition}[Groupe de type fini]
Si G est engendré par un singleton, on dit que G est monogène. \\
Un groupe monogène fini est dit cyclique. \\Si il existe une partie finie $S\subseteq G$ qui engendre G, on dit que G est de type fini.
\end{definition}

\begin{definition}[Ordre d'un élément]
\begin{itemize}[label=$\bullet$]
\item Si $\langle x \rangle$ est infini, on dit que $x$ est d'ordre infini.
\item Si $\langle x \rangle$ est fini, on dit que $x$ est d'ordre $\lvert\langle x \rangle\rvert$
\end{itemize}
Si $x^{n}=e$ alors $o(x)|n$
\end{definition}

\section{morphismes de groupes}
\begin{definition}[Morphisme de groupe]
Soit $(G,\ast), (H,\cdot)$ deux groupes. Un morphisme de groupes de G dans H est une application
{\color{white}-}\\
$\fctsarg{f}{G}{H}$ tel que $\forall x,y \in G, f(x \ast y) = f(x) \cdot f(y)$
\end{definition}

\begin{exo}
\begin{enumerate}
\item $f(e_{G})=e_{H}$
\item $f^{-1}(x)=f(x^{-1})$
\item $\forall n \in \N~,~ f^{n}(x)=f(x^{n})$
\item Si $K < G,$ alors $f(K) < H$
\item Si $K < H,$ alors $f^{-1}(K) < G$
\end{enumerate}
\end{exo}

\begin{exemple}
\begin{enumerate}
    \item $\fctsarg{\epsilon}{\sn}{\{-1,1\}}$ 
    \item $\fctsarg{det}{\gl{n}{\R}}{\R^*}$
    \item $\fctsarg{exp}{\C}{\C^*}$
    \item Mais $\fctsarg{exp}{\mat{2}{R}}{(\gl{2}{\R},\times)}$
\end{enumerate}
\end{exemple}

\subsection{Isomorphismes}

\begin{definition}[Isomorphisme]
\begin{enumerate}
    \item Un isomorphisme de $G$ dans $H$ est un morphisme de groupes bijectif.
    \item $G$ et $H$ sont isomorphe ssi il existe un isomorphisme entre les deux.
\end{enumerate}
\end{definition}

\begin{exo}
    Si $f$ est un isomorphisme alors $f^{-1}$ aussi
\end{exo}

\begin{exo}
\begin{enumerate}
    \item $\Z /4\Z$ et $\Z/2\Z \times \Z/2\Z$ ne sont pas isomorphe
    \item $\Z/6\Z$ et $\mathbb{S}_n$ ne sont pas isomorphe (car l'un est abélien et l'autre non).
    \item $\Z/mn\Z$ et $\Z/m\Z \times \znz$ sont isomorphe ssi $m\wedge n = 1$
\end{enumerate}
\end{exo}

\begin{definition}[Automorphisme]
    Un automorphisme est un isomorphisme d'un groupe $G$ dans lui-même. L'ensemble des automorphismes de G se note $Aut(G)$.
\end{definition}

\begin{exo}
    Montrer que $Aut(G) < \mathbb{S}_G$, où $\mathbb{S}_{G}$ désigne l'ensemble des bijections de $G$ dans lui même
\end{exo}

\begin{exo}
    $\forall g\in G$, on note $\fonction{\sigma_g}{G}{G}{x}{gxg^{-1}}$ (automorphisme intérieur associé à g), montrer que $\sigma_g \in Aut(G)$
\end{exo}

\begin{exo}
    On note $Int(G)$ l'ensemble des automorphismes intérieurs de $G$, montrer que $Int(G) < Aut(G)$
\end{exo}

\begin{theoreme}[Théorème de Cayley]
    Tout groupe G est isomorphe à un sous-groupe de $\mathbb{S}_G$. En particulier, si $|G| = n$, alors $G$ est isomorphe à un sous-groupe de $\sn$.
\end{theoreme}

\begin{preuve}
    Pour tout $g\in G$, on pose $\fonction{\tau_g}{G}{G}{x}{gx}$ $\tau_g$ est une bijection de $G$ dans $G$. Notons $T_G := \{\tau_g,g\in G\} \subseteq \mathbb{S}_G$. \par
    \noindent Vérifions que:  \par
    
    \begin{enumerate}
    \item $T_G < \mathbb{S}_G$
    \item $G$ est isomorphe à $T_G$
    \end{enumerate}

\noindent\underline{Preuve de 1:}
\begin{itemize}[label = $\bullet$]
\item $Id_G = \tau_e \in T_G (T_G \neq \emptyset)$
\item  $\forall g_1,g_2 \in G,\forall x \in G, \tau_{g_1 g_2}(x) = g_1 g_2 x = g_1(g_2 x) = \tau_{g_1}(\tau_{g_2}(x))$, donc on a bien $\tau_{g_1 g_2} = \tau_{g_1} \tau_{g_2}$
\item $\forall g \in G, \tau_{g^{-1}} \circ \tau_g = \tau_g \circ \tau_{g^{-1}} = Id_G$ Donc $(\tau_g)^{-1} = \tau_{g^{-1}} \in T_G$
\end{itemize}

\noindent\underline{Preuve de 2:}

    Notons $\fonction{\phi}{G}{T_G}{g}{\tau_g}$
    Alors $\phi$ est un morphisme (d'après la preuve de 1) $\phi$ est immédiatement surjectif, mais il est également injectif :\\
Soit $g\in G$ tel que $\tau_g = Id_G$. Alors $\forall x \in G, gx =  x$. Si on prend $x =  e_G$, on obtient $g = e_G$. Donc $\ker(\phi) = {e_G}$, et donc $\phi$ est injectif.
\end{preuve}

\section{Produits directs}

\begin{definition}[Produit direct]
    Le groupe "produit direct" de deux groupes $G_1, G_2$ est l'ensemble $G_1 \times G_2$ muni de la loi :\\
    $\fonction{\cdot}{(G_1 \times G_2)\times(G_1 \times G_2}{G_1 \times G_2}{((x_1 ,x_2),(y_1 ,y_2))}{(x_1 y_1 ,x_2 y_2)}$
\end{definition}

\begin{exo}
    vérifier que $G_1 \times G_2$ muni de cette loi est bien un groupe.
\end{exo}
\begin{definition}[Projections et injections canoniques]
\begin{enumerate}
    \item Projections canoniques $\fonction{p_i}{G_1 \times G_2}{G_i}{(x_1 ,x_2)}{x_i}$.
    \item Injections canoniques : $\fonction{q_1}{G_1}{G_1 \times G_2}{x_1}{(x_1 ,e_2)}$ et $\fonction{q_2}{G_2}{G_1 \times G_2}{x_2}{(e_1 ,x_2)}$
\end{enumerate}
\end{definition}

\remarque{$\im (q_i)$ est isomorphe à $G_i$. Ainsi $G1 \times G_2$ contient un sous-groupe isomorphe à $G_1$, de même pour $G_2$.}
    


\remarque{
$\forall ~x = (x_1 ,x_2) \in G_1 \times G_2,$ on a :\par
    $x = (p_1 (x), p_2 (x)) = (x_1 ,x_2) = (x_1 ,e_2)(e_1 ,x_2) = (e_1 ,x_2)(x_1 ,e_2) = q_1 (x_1)q_2 (x_2) = q_2(x_2)q_2(x_1)$}
    

\begin{theoreme}
    Un groupe $G$ est isomorphe au produit direct $G1 \times G_2$ ssi $G$ contient deux sous-groupes $H_1,H_2$ tel que :
    \begin{enumerate}
        \item $H_i$ est isomorphe à $G_i (i = 1,2)$
        \item $h_1 h_2 = h_2 h_1, \forall h_1\in H_1, \forall h_2\in H_2$
        \item $G = H_1 H_2$
        \item $H_1 \cap H_2 = \{e_G\}$
    \end{enumerate}
\end{theoreme}

\begin{preuve}
    \fbox{$\Rightarrow$} Supposons qu'il existe $\fctsarg{\phi}{G_1 \times G_2}{G}$ isomorphe.\par
    
    \begin{enumerate}
    \item On a que $ G_{1}\simeq \{G_{1},e_{2}\} \simeq \phi\left(\{G_{1},e_{2}\}\right):=H_{1} $ il suffit alors de remarquer que $H_{1}$ est un sous groupe de G. On construit de même $H_{2}$
    \item$\forall (h_{1},h_{2})\in H_{1}\times H_{2}$, on note $h'_{1}= (h_{1},e_{2})$ idem pour $h'_{2}$, on a alors : 
    \begin{center}
    $\dsp h_1 h_2 = \phi(h_1 'h_2 ') = \phi(h_2 'h_1 ') = h_2 h_1 $
    \end{center}
    
    \item $\forall x \in G, \exists!~x'=(h_{1},h_{2})\in G_{1}\times G_{2}$ tel que $\phi(x')=x$. On a alors :  \par
    \begin{center}
    $\dsp x = \phi(x') = \phi(h_{1}'h_{2}') = h_{1}h_{2} $
    \end{center}
    \item Immédiat
    \end{enumerate}
    \noindent\fbox{$\Leftarrow$} Construisons un isomorphisme de $G$ dans $G_1 \times G_2$\\
    Fait : $\forall g\in G, \exists! (h_1,h_2)\in H_1 \times H_2 $ tel que $g = h_1 h_2$\\
    En effet : l'existence vient de 3), l'unicité vient de 4) :$g = h_1 h_2 = k_1 k_2$ alors $(k_1)^{-1} h_1 = k_2 (h_2)^{-1}$. Comme $H_1 \cap H_2 = \{e\}$ on obtient $(k_1)^{-1} h_1 = k_2 (h_2)^{-1} = e_G \Rightarrow h_1 = k_1 $ et $h_2 = k_2$\\
    Notons $\fctsarg{\phi_1}{H_1}{G_1}$ et $\fctsarg{\phi_2}{H_2}{G_2}$ les isomorphismes données par 1).\\
    Posons $\fonction{\phi}{G}{G_1 \times G_2}{h_1 h_2}{(\phi_1 (h_1),\phi_2 (h_2))}$\\
    Mq $\phi$ est un morphisme $(\alpha)$, injectif $(\beta)$, surjectif $\gamma$\\
    $(\alpha): \phi(h_1 h_2 h_1 'h_2 ') =\phi(h_1 h_1 'h_2 h_2') = (\phi_1(h_1 h_1'),\phi_2(h_2 h_2 ')) = (\phi_1(h_1),\phi_1(h_1 '),\phi_2(h_2)\phi_2(h_2 ')) = (\phi_1 (h_1),\phi_2 (h_2)) (\phi_1 (h_1 '),\phi_2(h_2 ')) = \phi(g)\phi(g')$\\
    $(\beta):$ Soit $x = h_1 h_2$ tel que $\phi(x) = (\phi_1 (h_1),\phi_2 (h_2)) = (e_1,e_2)$\\
    Alors $\phi_1(h_1) = e_1$ et $\phi_2 (h_2) = e_2 \Rightarrow h_1 = h_2 = e_G$\\
    $(\gamma):$ Soit $x = (x_1,x_2) \in G_1 \times G_2$, soit $(h_1,h_2)$ tel que $\phi_i (h_i) = x_i$, alors $x = (\phi_1 (h_1),\phi_2 (h_2)) = \phi(h_1,h_2)$, cela montre la surjectivité de $\phi$.
\end{preuve}

\begin{exemple}
$(\Z / 2^{\alpha}\Z,+,\times)$ est anneau. On note $(\Z / 2^{\alpha}\Z)^{\times}$ l'ensemble des éléments inversibles de l'anneau (pour la loi $\times$). Si $\alpha \ge 3$, $(\Z / 2^{\alpha}\Z)^{\times}$ est isomorphe à $(\Z /2^{\alpha -2}\Z)\times(\Z / 2\Z)$
\end{exemple}

\chapter{Classes modulo un sous-groupe, sous-groupes distingués}


\section{Classes à droite, classes à gauche}

Soit $H < G$. On définit $x\mathcal{R}_H y \iff xy^{-1} \in H$ et $x _{H}\mathcal{R} y \iff x^{-1}y \in H$

\begin{exemple}
\begin{enumerate}
    \item $\mathcal{R}_H$ et $_{H}\mathcal{R}$ définissent deux relations d'équivalences
\item La classe d'équivalence de $x$ pour $\mathcal{R}_H$ est $Hx$ appelée classe à droite de $x$ modulo $H$, idem pour $_{H}\mathcal{R}$
\end{enumerate}
\end{exemple}

\begin{exemple}
	On se place dans $\mathcal{S}_3$, on pose $\sigma=(1,2,3)$ et $\tau=(1,2)$, on a alors $\mathbb{S}_3 = \{e,\sigma,\sigma^2,\tau,\tau\sigma,\sigma\tau\}$.\par Pour $H = \{e,\tau \}$, on a :\par
    $H\sigma = \{\sigma, \tau\sigma\}, H\sigma^2 = \{\sigma^2,\tau\sigma^2 (=\sigma\tau)\}$\par
    $\sigma H = \{\sigma,\sigma\tau\},\sigma^2 H = \{\sigma^2,\sigma^2 \tau (=\tau \sigma)\}$\par
    donc $\sigma H \neq H \sigma$.
\end{exemple}

\begin{exemple}
    Si $G$ est abélien, on a $xH = Hx,\forall x \in G$.
\end{exemple}

\remarque
$\forall g \in G, \fonction{\tau_g}{G}{G}{x}{gx}$ est une bijection. En particulier, $\tau_g|_H$ est une bijection de $H$ sur $gH$. De même, $\fonction{\rho_g}{G}{G}{x}{xg}$, alors $\rho_g|_H$ est une bijection de $H$ sur $Hg$.

\remarque 
Soit $\{e\}\cup\{x_i,i\in I\}$ un système de représentants des classes à gauche modulo H. On a alors $G = H \sqcup \displaystyle\bigsqcup_{i \in I} x_i H$ (union disjointe).

\remarque
L'application $\fctsarg{}{x_i H}{H (x_i)^{-1}}$ est une bijection de l'ensemble des classes à gauche sur l'ensemble des classes à droite.

\begin{definition}[Indice de H dans G]
    L'indice de $H$ dans $G$ est le cardinal (fini ou infini) de l'ensemble des classes à gauche (= cardinal de l'ensemble des classes à droite), il est noté $[G:H]$
\end{definition}

On en déduit le théorème de Lagrange :
\begin{theoreme}[Théorème de Lagrange]
    Soit $G$ un groupe fini et $H < G$. Alors :
    \begin{enumerate}
        \item $|G| = |H|[G:H]$
        \item $\forall x \in G, o(x)$ $|$ $|G|$
    \end{enumerate}
\end{theoreme}

\section{Sous-groupes distingués}

\begin{definition}
    Soit $G$ un groupe fini, $H < G$ est dit distingué (ou normal) dans $G$ ssi $\forall x \in G, xH = Hx$.\\
    Le cas échéant on note : $H \triangleleft G$
\end{definition}

\begin{definition}
    Un groupe $G$ est dit simple ssi ses seuls sous-groupes distingué sont $\{e\}$ et $G$.
\end{definition}

\remarque
Si $G$ est abélien, tout $H < G$ est distingué.

\begin{exemple}
    Soit $H < G$. Alors $H \triangleleft G \iff \forall g \in G, gHg^{-1} = H$
\end{exemple}

\begin{prop}
    Soit $H\backslash G$ l'ensemble des classes à gauche modulo $H$.\\
    L'application $\fctsarg{}{(xH,yH)}{xyH}$ est bien définie ssi $H \triangleleft G$.\\
    Idem pour les classes à droites $G/H$.
\end{prop}

\begin{preuve}
$\Rightarrow :$ Soit $h \in H,y \in G$, l'application est bien définie, donc $egH = hgH$ donc $yH = hgH$ donc $H = y^{-1}hyH$, donc $y^{-1}hy \in H$.\\
$\Leftarrow : $ Si $x,x' \in G$ tel que $xH = x'H$, et si $y,y' \in G$ tel que $yH = y'H$, alors on a $h,h' \in H$ vérifiant : $x' = xh$ et $y' = yh'$. Donc $x'y' = xy y^{-1}hyh'$, avec $y^{-1}hyh' \in H$ car $H \triangleleft G$. Donc $x'y'H \subseteq xyH$, par symétrie on a $\supseteq$
\end{preuve}

\begin{theoreme}[Groupe quotient]
    Soit $G$ un groupe, $H \triangleleft G$. On note $\bar{x}$ la classe de $x$ modulo $H$, $\displaystyle\frac{G}{H}$ l'ensemble des classes modulo $H$. Alors : \par
    \begin{enumerate}
        \item L'application $\fonction{\ast}{(\displaystyle\frac{G}{H})\times (\displaystyle\frac{G}{H})}{\displaystyle\frac{G}{H}}{(\bar{x},\bar{y})}{\bar{x} \ast \bar{y} := \bar{xy}}$ munit $\displaystyle\frac{G}{H}$ d'une structure de groupe tel que $\bar{e} = H$ est l'élément neutre.
        \item En particulier, l'application $\fctsarg{\pi}{G}{\displaystyle\frac{G}{H}}$ est un morphisme de groupes de noyau $H$.
    \end{enumerate}
\end{theoreme}

\subsection{Sous-groupes distinguées et noyaux}

\begin{prop}
    Si $\fctsarg{\phi}{G}{G'}$ un morphisme, alors $\ker(\phi) \triangleleft G$.
\end{prop}

\begin{preuve}
    Si $h \in \ker(\phi), g \in G, \phi(ghg^{-1}) = \phi(g)\phi(h)\phi(g)^{-1} = \phi(g)\phi(g)^{-1} = e_{G'}$, donc $ghg^{-1} \in \ker(\phi)$.
\end{preuve}

\begin{theoreme}[Groupes distingués et morphismes]
    Soit $G$ un groupe. Alors $H \triangleleft G \,$ ssi $\, \exists G'$ groupe, $\exists \fctsarg{\phi}{G}{G'}$ morphisme tel que $H = \ker(\phi)$
\end{theoreme}

\begin{exemple}
    \begin{enumerate}
        \item $\fctsarg{\varepsilon}{\mathbb{S}_n}{-1,1} (signature)$, alors $A_n := \ker(\varepsilon) \triangleleft \mathbb{S}_n$
        \item $\fctsarg{det}{\mathbb{GL}_n(\R)}{\R^*}$, alors $\mathbb{SL}_n(\R) := \ker(det) \triangleleft \mathbb{GL}_n(\R)$
    \end{enumerate}
\end{exemple}

\begin{theoreme}[Premier théorème d'isomorphisme]
    Soit $\fctsarg{\phi}{G}{G'}$ un morphisme de groupe. Alors, $G/\ker(\phi)$ est isomorphe à $Im(\phi)$.
\end{theoreme}

\chapter{\texorpdfstring{Étude de $\znz$, de $\sn$, de $\mathbb{D}_{n}$}{Etude de Z/nZ, Sn, Dn}}

\section{J'ai pas le nom...}




\subsection{Autres exemples de sous groupes normaux}

\begin{itemize}
\item j'ai pas le premier...
\item Le centre d'un groupe $\dsp Z(G)=\{g\in G,~gx=gx~\forall x\in G\}$ est un sous groupe normal de G. (preuve en exercice (feuille 3)). $Z(G)$ est en fait \underline{caractéristique} c'est à dire qu'il est invariant par tout automorphisme intérieur
\item Le groupe \underline{dérivé} de G est le sous-groupe (noté D(G)) qui est engendré par les commutateurs de G c'est à dire les éléments de la forme $[a,b]=aba^{-1}b^{-1}$ est aussi un sous-groupe normal.
\end{itemize}

\begin{exemple}
    \begin{enumerate}
        \item Si G est abélien alors $Z(G)=G$
        \item Si $n\leq3$ alors $Z(S_{n})=\{e\}$
    \end{enumerate}
\end{exemple}

\begin{preuve}
    Preuve du deuxième point:\\
    Soit $\sigma \in \sn$ avec $\sigma \neq e$.\\
    Soit alors $i\in \llbracket 1,n \rrbracket$ tel que l'on ait $\sigma(i):=j\neq i$\\
    Soit enfin $k\in \llbracket 1,n \rrbracket \setminus \{i,j\}$, on pose $\tau = (j,k)$.\\
    On à bien $\sigma\tau\neq\tau\sigma$, car $\sigma\tau(i)=j\neq\tau\sigma(j)=k$
\end{preuve}


\begin{exo}
    \begin{itemize}
        \item $D(G)\vartriangleleft G$ et $G/D(G)$ est abélien
        \item Soit $H \vartriangleleft$ alors $G/H$ est abélien$\Leftrightarrow D(G)<H$
        \item D(G) est un sous groupe caractéristique de G
        \item $\forall n \leq3 ~~D(S_{n})=A_{n}$ ou $A_{n}$ est le groupe alterné, désigne les permutations de signature paire
    \end{itemize}
\end{exo}

\begin{definition}[Normalisateur d'un sous-groupe]
Soit $H<G$, on note$N_{G}(H) = \{g\in G , ~gH=Hg\}$, on l'appelle le normalisateur de H dans G
\end{definition}

\begin{exo}
Mq $H\triangleleft N_{g}(H)$ et que $N_{g}(H)<G$
\end{exo}

\begin{exemple}
Dans $A_{4}$ \par
Soit $H =\{ e,(1,2),(3,4) \}<A_{4}$, $|H|=2$.  \par
O a $H<D(A_{4})$ et $H\triangleleft D(A_{4})$ car $\dfrac{|D(A_{4})|}{|H|}=2$.  \par
Verifier que $N_{A_{4}}(H)=D(A_{4})$ :  \par

Soit $N = N_{A_{4}}(H)$ pour simplifier. On sait que $D(A_{4})<N$ donc $|D(A_{4}|=4$ divise $|N|$ donc $|N|\in \{ 4,8,12 \}$, mais vu $N<A_{4}$, $|N|$ divise 12, donc $|N|=4$ où $|N|=12$. Mais $N\neq A_{4}$ car $(1,2,3)H(1,2,3)\neq H$  \par
\end{exemple}


\section{Groupes Monogènes, cycliques, symétriques, diédraux}

\subsection{Groupes Monogènes}

\begin{definition}[Groupe monogène]
Un groupe G est dit \textit{monogène} s'il est engendré par une unique élément
\end{definition}

\begin{theoreme}
Soit G un groupe monogène alors : \par
\begin{itemize}
\item Ou bien G est isomorphe à $\Z$
\item Ou bien G est isomorphe à $\Z / n\Z$ pour un certain $n \in \N$
\end{itemize}
\end{theoreme}

\begin{preuve}

\noindent Soit $G = \langle x \rangle$ et soit $\fonction{\psi}{\Z}{G}{k}{x^{k}}$. $\psi$ est un morphisme de groupe, il est surjectif. \par
\quad Si il est injectif on à bien $G\simeq \Z$. \par
Sinon, il existe $n\in \N$ tq $\ker \psi = n\Z$  \par

\begin{center}
\begin{tikzpicture}
\node (F) at (0, 0) {$\Z$};
\node (G) at (3, 0) {G};
\node (F/G) at (0, -2) {$\Z/n\Z$};
\draw [->] (F) to node [auto] {$\psi$} (G);
\draw [->] (F) to node [auto, swap] {$\pi$} (F/G);
\draw [->] (F/G) to node [auto, swap] {$\tilde{\psi}$} (G);
\end{tikzpicture}
\end{center}
Et d'après le premier théorème d'isomorphisme, il existe un \underline{isomorphisme} de groupe\\$\dsp\Tilde{\psi} : \Z/n\Z \longrightarrow Im(\psi)=G$ tel que le diagramme ci-dessus commute.

\end{preuve}

\begin{prop}
Tout groupe fini d'ordre p avec p premier est cyclique
\end{prop}

\begin{preuve}
Il suffit d'utiliser le théorème de Lagrange.
\end{preuve}


\subsection{Sous-groupes d'un groupe monogène}

\begin{prop}
\begin{enumerate}
\item Tout sous-groupe non trivial d'un groupe monogène infini est infini
\item Tout sous groupe d'un groupe cyclique est monogène et cyclique
\end{enumerate}
\end{prop}

\begin{preuve}
\begin{enumerate}
\item Ici $G\simeq \Z$, donc tout $H<G$ est isomorphe à un sous-groupe de $\Z$ ie. un groupe de la forme $n\Z$ pour $n \neq 0$, donc H est infini
\item On reprend le diagramme :

\begin{center}
\begin{tikzpicture}
\node (F) at (0, 0) {$\Z$};
\node (G) at (3, 0) {G};
\node (F/G) at (0, -2) {$\Z/n\Z$};
\draw [->] (F) to node [auto] {$\psi$} (G);
\draw [->] (F) to node [auto, swap] {$\pi$} (F/G);
\draw [->] (F/G) to node [auto, swap] {$\tilde{\psi}$} (G);
\end{tikzpicture}
\end{center}

Soit $ K < G= \Z/n\Z$ on à $K=\pi(\pi^{-1}(K))$ car $\pi$ est surjective. Comme $\pi^{-1}(K)$ est un sous groupe de $\Z$ il existe $k>0$ tq $\pi^{-1}(K)=k\Z$.\\
Alors $K = \pi(k\Z)$ est le sous-groupe de $\Z/n\Z$ engendré par $\pi(k)$, $K$ est donc monogène et fini

\end{enumerate}
\end{preuve}

\remarque{Si on reprend la preuve précédente on à $\pi^{-1}(0) = n\Z \subset \pi^{-1}(K)= k\Z$.\\ Ainsi, $n\Z \subset k\Z$ et donc $k|n$. Par conséquent, pour tout sous-groupe K de $\znz$, il existe un diviseur $k$ de $n$ tel que $\pi(k)$ engendre K, l'ordre de $\pi(k)$ étant $\dfrac{n}{k}$, on a $|K|=\dfrac{n}{k}$ en particulier ce diviseur est unique on à donc le théorème suivant.}

\begin{theoreme}
Soit $G = \langle x \rangle$ un groupe cyclique d'ordre n alors :\\
Pour tout diviseur $d$ de $n$, il existe un unique sous groupe d'ordre d de G et ce sous groupe est engendré par $x^{n/d}$
\end{theoreme}

\begin{prop}

Soit G un groupe non trivial alors :

\begin{center}
$G$ n'a pas de d'autres sous-groupes que $G$ et $\{ e \}$ $\Longleftrightarrow  $ G est cyclique d'ordre $p$ premier
\end{center} 
\end{prop}

\begin{preuve}
\fbox{$\Leftarrow$} évident par Lagrange


\noindent \fbox{$\Rightarrow$} Soit$x \in G\setminus\{e\}$ alors $\langle x>=G$ par hypothèse. Si G était infini, il posséderait des sous-groupes non triviaux de type $n\Z$, donc G est fini. Comme il n'a pas d'autres sous-groupes que $\{ e\}$ et G on a forcément $|G|=p$ premier par le théorème précédent.
\end{preuve}

\begin{theoreme}
	Soit $G$ un groupe monogène : $G=\langle x \rangle$
	\begin{enumerate}
		\item Si $G$ est infini, alors les seuls générateurs de $G$ sont $x$ et $x^{-1}$ 
		\item Si $G$ est fini (il est cyclique d'ordre $n$) alors l'ensemble de ses générateurs est donné par $\{x^k : k\in \mathbb{Z}, k \wedge n =1\}$
	\end{enumerate}
\end{theoreme}

\begin{preuve}
	\begin{enumerate}
	\item	Soit $\psi : k \in \mathbb{Z} \to x^k \in G \,$ (vue précédemment) qui est un isomorphisme de groupes. En particulier, $\psi$ échange les générateurs. Comme les seuls générateurs de $\mathbb{Z}$ sont $1$ et $-1$, on conclut. 
	\item	Soit $k \in \mathbb{Z}$, alors : 
		\begin{equation*} 
		\begin{split}
			G=\langle x \rangle  &  \iff \exists m \in \mathbb{Z}, x^{km}=x \\
 			& \iff \exists m \in \mathbb{Z}, n | km-1 \\
			&  \iff \exists (m,q)\in \mathbb{Z}, km-nq=1 \\
			& \iff pgcd(k,n)=1
		\end{split}
\end{equation*}
	
	\end{enumerate}
	

\end{preuve}

\begin{exo}
	L'ensemble des générateurs de $G \simeq \mathbb{Z}/n\mathbb{Z}$ est aussi égal à $\{\bar{k} \in \mathbb{Z}/n\mathbb{Z} : 0 \leq k \leq n-1, k \wedge n =1\}$
\end{exo}

\begin{definition}[Fonction d'Euler]
	La fonction d'Euler est la fonction $\varphi : \mathbb{N}^* \rightarrow \mathbb{N}^*$ telle que :
\begin{itemize}
\item $\varphi (1) = 1$ 
\item $\varphi (n) = | \{k \in \mathbb{N} : 1 \leq k \leq n, k \wedge n =1 \}|$
		
\end{itemize}
\end{definition}




\section{\texorpdfstring{Anneau $\mathbb{Z}/n\mathbb{Z}$}{Anneau Z/nZ}}

On rappelle que les opérations d'addition et de multiplication sont bien définies sur $\mathbb{Z}/n\mathbb{Z}$ (pas de dépendance des représentants) et que cet anneau est unitaire. 

\begin{definition}[Inverse modulo $n$]
On dit que $\bar{k}\in \mathbb{Z}/n\mathbb{Z}$ est inversible s'il existe $\bar{m} \in \mathbb{Z}/n\mathbb{Z}$ tel que $\bar{k}\bar{m}=\bar{1}$
\end{definition}

\begin{prop}
	Soit $n \geq 2$. Les éléments inversibles de $\mathbb{Z}/n\mathbb{Z}$ sont exactements les générateurs de $\left (\mathbb{Z}/n\mathbb{Z}, + \right)$ \\
	L'ensemble des éléments inversibles est alors un groupe abélien fini d'ordre $\varphi (n)$.
\end{prop}
\begin{preuve}
	Utiliser la caractérisation précédente avec Bézout.
\end{preuve}

\section{\texorpdfstring{Produits directs de groupes cycliques, calcul de $\varphi (n)$}{Produits, directs de groupes cycliques, indicatrice d'Euler}}
On considère le morphisme d'anneaux unitaires : $$f : k\in \mathbb{Z} \to (\bar{k}, \bar{\bar{k}}) \in \mathbb{Z}/n\mathbb{Z} \times \mathbb{Z}/m\mathbb{Z}$$

\begin{theoreme}
	Le morphisme d'anneaux unitaires $f$ induit par passage au quotient par son noyau un isomorphisme d'anneaux unitaires 
	$\bar{f} : \mathbb{Z}/nm\mathbb{Z} \rightarrow \mathbb{Z}/n\mathbb{Z} \times \mathbb{Z}/m\mathbb{Z}\; \; \text{si et seulement si}\; m\wedge n =1$
\end{theoreme}

\begin{preuve}
	Il faut vérifier $\bar{f}$ est bijective ssi $m \wedge n=1$ : \par
	\begin{equation*} 
	\begin{split}
		f \text{ est surjective}  & \iff | Im(f) | =mn \\
 		& \iff |\mathbb{Z}/\ker(f)| = mn \; \textit{(grâce au théorème d'isomorphisme)} \\ 
		& \iff \ker(f)= mn\mathbb{Z} \\ 
		& \iff m\mathbb{Z} \cap n\mathbb{Z} = mn\mathbb{Z} \\
		& \iff m \wedge n =1
	\end{split}
\end{equation*}
\end{preuve}

\begin{prop}
	Si $m \wedge n =1$, alors $\varphi (nm) = \varphi (n) \varphi (m)$
\end{prop}

\begin{theoreme}
	Soit $n=p_1^{\alpha_1} \cdots p_k^{\alpha_k}$, décomposé en facteur premiers. Alors : $$\varphi (n) = n \left (1-\frac{1}{p_1} \right )\times \cdots  \times \left (1-\frac{1}{p_k} \right ) $$
\end{theoreme}

\begin{preuve}
Il nous suffit de calculer $\varphi (p^\alpha)$ pour $p$ premier et $\alpha \geq 1$. On a : 
\begin{equation*} \label{eq1}
	\begin{split}
	\varphi (p^\alpha) & = | \{ k \in \{1, \cdots , p^\alpha \} : k \wedge p^\alpha =1 \}|  \\
	& = | \{ 1, \cdots , p^\alpha \} \backslash \{ p, 2p, \cdots , p^{\alpha -1}p \} | \\
	& =p^{\alpha}-p^{\alpha -1}
	\end{split}
\end{equation*}
\end{preuve}

\section{Structure des groupes abéliens finis (admis)}
Référence : Livre de F. Ulmer "Théorie des groupes" chapitre 12


Soit $G$ un groupe fini abélien d'ordre $N$. Il existe une décomposition unique $N=d_1 \cdots d_n$ avec $d_n \geq 2$ et $d_{i+1} | d_i$ telle que : $$G \simeq \mathbb{Z}/d_1\mathbb{Z} \times \dots \times \mathbb{Z}/d_n\mathbb{Z}$$

\begin{exemple}
	On peut lister, à isomorphisme près, tous les groupes abéliens d'ordre $72=3^2 \times 2^3$ avec les séquences suivantes : $(3^2 \times 2^2, 2) , (3\times 2, 3\times 2, 2), (3\times 2^3, 3), (2^2\times 3, 2\times 3), (3^2\times 2,2,2)$
\end{exemple}

\section{Groupes symétriques}
On note $\sn$  l'ensemble des permutations de $\{1, 2, ..., n\}$ que l'on munit de la loi de composition : c'est un groupe d'ordre $n!$

\subsection{Support et Orbite}
\begin{definition}[Support]
Le support de $\sigma \in \sn$ est l'ensemble $\{i\in\unn\; ; \; \sigma(i)\neq i\}$
\end{definition}

\begin{exo}
Soit $\sigma\in\sn$. Montrer que
\begin{itemize}[label=$\bullet$]
\item $\sigma$ et $\sigma^{-1}$ ont le même support
\item Deux permutations dont les supports sont disjoints commutent
\end{itemize}
\end{exo}

\begin{definition}[Orbite]
Soit $\sigma\in\sn$. On définit la relation d'équivalence sur $\unn$ : \[i\mathcal{R}j \Longleftrightarrow \exists r\in\Z \;|\;\sigma^r(i)=j.\]
La classe de $i$ est notée $\Omega(i) = \{\sigma^r(i), r\in\Z\}$ et est appelée $\sigma$-orbite de $i$.
\end{definition}

\subsection{Notion de cycle}
\begin{definition}[r-cycle]
$\sigma \in\sn$ est un r-cycle si il existe $j_1, ..., j_r$ dans $\unn$ tq $\sigma(j_1)=j_2, ..., \sigma(j_{r-1})=j_r, \sigma(j_r)=j_1$, et si pour $k\notin\{j_1,...,j_r\}, \sigma(k)=k$.\\Alors le support de $\sigma$ est $\{j_1,...,j_r\}$. On notera $\sigma = (j_1, ..., j_r)$
\end{definition}

\begin{definition}[Transposition et permutation circulaire]
\begin{enumerate}
\item Un 2-cycle est appelé transposition
\item le n-cycle $(1, ..., n)$ est appelé permutation circulaire
\end{enumerate}
\end{definition}

\begin{exemple}
Si $\dsp\sigma_0=\left(\begin{array}{cccccc}
1&2&3&4&5&6\\3&1&2&6&5&4
\end{array}\right)
\begin{array}{cl}
\leftarrow&i\\\leftarrow&\sigma_0(i)
\end{array}$\\
alors $\sigma_0=(1,3,2)(4,6)$.
\end{exemple}

\begin{theoreme}
Toute permutation $\sigma\in\sn\setminus\{e\}$ se décompose sous la forme $\sigma=\gamma_1\circ\gamma_2\circ ... \circ\gamma_s$ où $s\in\N^*$, et où les $\gamma_i$ sont des cycles différents de $e$ dont les supports sont disjoints deux à deux. Cette décomposition est unique à l'ordre près des facteurs.
\end{theoreme}

\begin{exo}
\begin{enumerate}
\item Montrer que l'ordre de $\sigma$ est égal au ppcm des longueurs des cycles $\gamma_1, ..., \gamma_s$.
\item Calculer $\sigma_0^{1000}$.
\end{enumerate}
\end{exo}

\subsection{Formules importantes}
\begin{prop}
Pour tout $\tau\in\sn$, $\tau(j_1,...,j_r)\tau^{-1}=(\tau(j_1),...,\tau(j_r))$.
\end{prop}
\begin{prop}
On a  : $(j_1,...,j_r)=(j_1,j_2)(j_2,j_3)...(j_{r-1},j_r)$
\end{prop}

\noindent\underline{\textbf{Cas particulier :}} $(a,b,c)=(a,b)(b,c)$

\noindent\underline{\textbf{Applications de ces deux propriétés :}}
\begin{enumerate}
\item Deux r-cycles de $\sn$ sont conjugués dans $\sn$
\item $(1, i)(1,j)(1,i)=(1,i)(1,j)(1,i)^{-1}=(i,j)$
\item $\sn$ est engendré par les transpositions du type $(j, j+1)$ où $j\in\{1,...,n-1\}$\\
preuve : laissée en exercice au lecteur, l'idée est de montrer que $(i,j)$ est un produit de transpositions du type $(k,k+1)$ par récurrence sur $j-1$ en utilisant $(i,j)=(j-1,j)(i,j-1)(j-1,j)$
\item $\sn$ est engendré par $(1,2)$ et $\eta=(1,2,...,n)$\\
preuve : $\eta^i(1,2)\eta^{-i}=(i+1,i+2)$
\end{enumerate}

\subsection{Générateurs}
Soit $n \geq 2$.

\begin{theoreme}
\begin{enumerate}
\item $\sn$ est engendré par les transpositions
\item $\sn$ est engendré par les transpositions du type $(1,j)$ où $j\in\{2,...,n\}$
\end{enumerate}
\end{theoreme}

\subsection{Centre}
\begin{theoreme}
$Z(\sn)=\{e\}$ pour $n=1$ et $n\geq3$.
\end{theoreme}

\subsection{Signature}
\begin{definition}[Signature]
Soit $\sigma\in\sn$. On pose $\epsilon(\sigma)=(-1)^{n-t}$ où $t$ est le nombre de $\sigma$-orbites différentes.
\end{definition}

\begin{exemple}
\begin{itemize}[label=$\bullet$]
\item $\sigma=e$ : on a $\sigma(i)=i$ pour tout $i\in\unn$, chaque point est une orbite donc $t=n$ et $\sign=1$
\item $\sigma=(1,2)$ : ici il y a $n-2$ éléments fixés qui donnent chacun une orbite, et $\{1,2\}$ est une autre orbite donc $\sign=-1$.
\item $\sigma=(1,...,r)$ : $\sign=(-1)^{r-1}$
\end{itemize}
\end{exemple}

\begin{prop}
Soit $\sigma\in\sn$ où $n\geq2$. Alors $\epsilon(\sigma\circ\tau)=(-1)\times\sign$ pour toute transposition $\tau\in\sn$.\\
En particulier, si $\sigma$ est un produit de $k$ transpositions, on a $\sign=(-1)^k$.
\end{prop}

\remarque{Ainsi, la parité du nombre de transpositions nécessaires pour décomposer $\sigma$ ne dépend que de $\sigma$.}

\begin{theoreme}
Si $n\geq2$, $\epsilon:\sn\longrightarrow\{1,-1\}$ est un morphisme de groupes surjectif.
\end{theoreme}
\begin{preuve}
Soient $\sigma,\sigma'\in\sn$. On décompose $\sigma=\tau_1\circ...\circ\tau_k$ et $\sigma'=\tau_1'\circ...\circ\tau_{k'}'$ en produits de transpositions. \\Alors $\epsilon(\sigma\circ\sigma')=(-1)^{k+k'}=\sign\times\epsilon(\sigma')$.
\end{preuve}

\begin{definition}[Groupe alterné]
Soit $n\geq2$. $\an$ est le noyau de $\epsilon$, on le nomme groupe alterné.
\end{definition}

\par
\par

\remarque{C'est un sous groupe distingué de $\sn$ d'indice 2, car le noyau d'un morphisme}


\remarque{Si $\tau$ est une transposition, $(\tau\an)\cap\an=\emptyset$, d'où $\sn=(\tau\an)\sqcup\an$.}
\begin{theoreme}
\begin{enumerate}
\item Si $n\geq3$, $\an$ est engendré par les 3-cycles.
\item Si $n\geq5$, deux 3-cycles sont conjugués dans $\an$ 
\item Si $n\geq2$, alors $D(\sn)=\an$, si $n\geq5$ alors $D(\an)=\an$.
\end{enumerate}
\end{theoreme}

\begin{preuve}
\begin{enumerate}
\item Soit $\sigma\in\an$, $\sigma$ est un produit d'un nombre pair de transpositions, or $(i,j)(j,k)=(i,j,k)$ et $(i,j)(k,l)=(i,j,k)(j,k,l)$.
\item Soient $(i,j,k),(i',j',k')$ deux 3-cycles. Il existe $\sigma\in\sn$ tel que $\sigma(i)=i', \sigma(j)=j', \sigma(k)=k'$. Alors $\sigma(i,j,k)\sigma^{-1}=(i',j',k')$. 
Sans perte de généralité, on peut supposer que $\sigma\in\an$, en effet $n\geq5$, donc il existe une transposition $\tau=(r,s)$ avec $r,s\notin\{i,j,k\}$, et on peut remplacer $\sigma$ par $\sigma\tau$.
\item $D(\an)\subset D(\sn)\subset\an$ car si $a,b\in\sn$, alors $\epsilon([a,b])=1$.\\
Montrons que si $n\geq5$, les 3-cycles, qui engendrent $\an$, sont des commutateurs (de $\an$).\\
Soit $\sigma=(i,j,k)$ un 3-cycle. $\sigma^2$ est aussi un 3-cycle donc d'après $2.$ les deux sont conjugués : il existe $\eta\in\an$ tel que $\sigma^2=\eta\sigma\eta^{-1}$ i.e. $\sigma=[\eta,\sigma]$.
\end{enumerate}
\end{preuve}

\noindent\underline{\textbf{Cas particuliers :}}
\begin{enumerate}
\item $D(\mathcal{A}_3)=\{e\}$
\item $D(\mathcal{A}_4)=\{e,(1,2)(3,4),(1,3)(2,4),(1,4)(2,3)\}$
\end{enumerate}

\begin{preuve}
\begin{enumerate}
\item $\mathcal{A}_3 = \langle (1,2,3) \rangle$ donc $\mathcal{A}_3\simeq \Z/3\Z$ est abélien
\item On note $V = \{e,(1,2)(3,4),(1,3)(2,4),(1,4)(2,3)\}$, c'est un sous groupe distingué de $\mathcal{A}_4$. Alors le groupe quotient $\mathcal{A}_4/V$ est d'ordre 3 donc isomorphe à $\Z/3\Z$ qui est abélien. Ainsi $D(\mathcal{A}_4)$ est un sous-groupe de $V$. Par le théorème de Lagrange, $D(\mathcal{A}_4)$ est de cardinal $1$, $2$, ou $4$. $\mathcal{A}_4$ n'est pas abélien donc ce n'est pas $1$. Si c'était $2$, $D(\mathcal{A}_4)$ serait de la forme $\{e, (i,j)(k,l)\}$ qui n'est pas distingué.
\end{enumerate}
\end{preuve}

\begin{prop}
Soit $\sigma \in \sn$,avec $n\neq2 $ alors : $\epsilon(\sigma\circ\tau) = (-1)\sign$
\end{prop}

\begin{preuve}
On veut étudier les orbites de $\sigma\circ\tau$. Seules les $\sigma$-orbites qui contiennent i ou j seront modifiées par $\tau$. $\tau$ agit comme l'identité sur les autres orbites.
\begin{itemize}[label=$\bullet$]
\item Premier cas : i et j appartiennent a la même orbite $O$: \par
	\noindent$O = \{i,\sigma(i),\sigma^2(i),...,\sigma^q(i)=j,\sigma^{q+1}(i),...,\sigma^{p-1}$, ou $p = |O|$. Vérifions alors que $\sigma\circ\tau$ sépare $O$ en deux $\sigma\circ\tau$-orbites :
	\begin{itemize}
		\item L'orbite de i par $\sigma\circ\tau$ noté $O_{i}$ vaut : \fcolorbox{red}{gray!5}{$O_{i}=\{ i,\sigma\circ\tau(i)=\sigma(j)=\sigma^{q+1}(i),...,\sigma^{p-1}(i) \}$}
		\item L'orbite de i par $\sigma\circ\tau$ noté $O_{j}$ vaut : \fcolorbox{blue}{gray!5}{$O_{j}=\{ j,\sigma\circ\tau(j)=\sigma(i),\cdots,\sigma^{q-1}(i) \}$}
			\begin{center}
				\scalebox{0.60}{%
    				\begin{minipage}{\textwidth}
        				\begin{tikzpicture}
\clip (-5,-5) rectangle (5,5);
\pgfmathsetmacro{\r}{4}
\draw (0,0) circle (\r);

%positionnement des points :
\node (i) at ({90:\r+0.3}) {i};
\node (si) at ({60:\r+0.4}) {$\sigma(i)$};
\node (ssi) at ({30:\r+0.5}) {$\sigma^{2}(i)$};
\node (spmui) at ({120:\r+0.5}) {$\sigma^{p-1}(i)$};
\node (j) at ({-70:\r+0.6}) {$j=\sigma^{q}(i)$};
\node (sj) at ({-105:\r+0.4}) {$\sigma(j)$};
\node (api) at ({0:\r+0.5}) {};
\node (avj) at ({-40:\r+0.5}) {};
\node (apj) at ({-135:\r+0.5}){};
\node (avi) at ({150:\r+0.5}) {};


% les flèches
\draw [line width = 0.75pt,->] (i) to [bend left = 20](si);
\draw [line width = 0.75pt,->] (si) to [bend left = 20](ssi);
\draw [line width = 0.75pt,->] (ssi) to [bend left = 20](api);
\draw [line width = 3pt, dash pattern = on 0pt off 12 pt, line cap=round](api) to [bend left =15] (avj);
\draw [line width = 0.75pt,->] (avj) to [bend left = 15](j);
\draw [line width = 0.75pt,->] (j) to [bend left = 20](sj);
\draw [line width = 0.75pt,->] (sj) to [bend left = 20](apj);
\draw [line width = 3pt, dash pattern = on 0pt off 12 pt, line cap=round](apj) to [bend left =30] (avi);
\draw [line width = 0.75pt,->] (avi) to [bend left = 20](spmui);
\draw [line width = 0.75pt,->] (spmui) to [bend left = 20](i);

%%%%%%%Les autres flèches%%%%%%%%%%%%%%%%%%%%%%ù

\node (i1) at ({90:\r-0.3}) {};
\node (si1) at ({60:\r-0.3}) {};
\node (ssi1) at ({30:\r-0.3}) {};
\node (spmui1) at ({120:\r-0.3}) {};
\node (j1) at ({-70:\r-0.3}) {};
\node (sj1) at ({-105:\r-0.3}) {};
\node (api1) at ({0:\r-0.3}) {};
\node (avj1) at ({-40:\r-0.3}) {};
\node (apj1) at ({-135:\r-0.3}){};
\node (avi1) at ({150:\r-0.3}) {};

\begin{scope}[red]
\draw [line width = 0.75pt,->] (i1) to [bend right = 20](si1);
\draw [line width = 0.75pt,->] (si1) to [bend right = 20](ssi1);
\draw [line width = 0.75pt,->] (ssi1) to [bend right = 20](api1);
\draw [line width = 3pt, dash pattern = on 0pt off 12 pt, line cap=round](api1) to [bend right =15] (avj1);
\draw [line width = 0.75pt,->] (avj1) to [bend right = 15](j1);
\draw [line width = 0.75pt,->] (j1) to [bend right = 20] node[auto,sloped]{$\sigma\circ\tau$}(i1);
\end{scope}

\begin{scope}[blue]
\draw [line width = 0.75pt,->] (sj1) to [bend right = 20](apj1);
\draw [line width = 3pt, dash pattern = on 0pt off 12 pt, line cap=round](apj1) to [bend left =30] (avi1);
\draw [line width = 0.75pt,->] (avi1) to [bend right = 20](spmui1);
\draw [line width = 0.75pt,->] (spmui1) to [bend left = 20] node[auto,sloped]{$\sigma\circ\tau$}(sj1);
\end{scope}

\end{tikzpicture}

    				\end{minipage}
				}
			\end{center}
	\end{itemize}
	On a bien montrer que $O_{i}\bigcap O_{j}=\emptyset$
\item Deuxième cas, i et j sont dans deux orbites différentes :\par 
\noindent On note $O' = \{j,\sigma(j),\cdots,\sigma^{q-1}(j) \}$  l'orbite de j sous $\sigma$ et $O = \{i,\sigma(i),\cdots,\sigma^{p-1}(i) \}$ l'orbite de i sous $\sigma$. A compléter...
\begin{center}
\scalebox{0.60}{%
    \begin{minipage}{\textwidth}
        \begin{flushleft}

\begin{tikzpicture}
\clip (-10,-7) rectangle (10,7);  % Agrandir le cadre
%%%%%%%%%%%%%%%%%%%%%%%%%%%%%%%%%%%%%%%%%%%%%%%%%%%%%%%%%%%%%%%%%%%%%%%%%%%%%%%%%%%
    % Dessiner l'arc de cercle (grande partie) en pointillé
    %\draw[line width = 3pt, dash pattern=on 0pt off 12pt, line cap=round] (zero) ++(130:\r+0.6) arc [start angle=130, end angle=-95, radius=\r+0.6];
%%%%%%%%%%%%%%%%%%%%%%%%%%%%%%%%%%%%%%%%%%%%%%%%%%%%%%%%%%%%%%%%%%%%%%%%%%%%%%%%%%%

\pgfmathsetmacro{\r}{4}

\begin{scope}[shift={(6,0)}] % Cercle de droite
	\draw (0,0)circle (\r);
    % Définition des nœuds :
    \node (avav) at ({315:\r+0.5}) {};
    \node (avant) at ({285:\r+0.5}) {};
    \node (spmdi) at ({255:\r + 0.5}) {$\sigma^{p-2}(i)$};
    \node (spmui) at ({225:\r + 0.6}) {$\sigma^{p-1}(i)$};
    \node (i) at ({195:\r + 0.3}) {i}; 
    \node (si) at ({165:\r + 0.4}) {$\sigma(i)$};
    \node (ssi) at ({135: \r +0.4}) {$\sigma^{2}(i)$};
    \node (apres) at ({105:\r+0.4}) {};
    \node (aprapr) at ({75:\r+0.4}) {};

    % Les flèches :
    \draw[line width = 0.75pt,->] (spmdi) to [bend left = 20] (spmui);
    \draw[line width = 0.75pt,->] (spmui) to [bend left = 20] (i);
    \draw[line width = 0.75pt,->] (i) to [bend left = 20] (si);
    \draw[line width = 0.75pt,->] (si) to [bend left = 20] (ssi);
    \draw[line width = 0.75pt,->] (ssi) to [bend left = 20] (apres);
	\draw[line width = 0.75pt,->] (avant) to [bend left = 20] (spmdi);    
    
    \draw [line width = 3pt, dash pattern = on 0pt off 12 pt, line cap=round](apres) to [bend left =15] (aprapr);
	\draw [line width = 3pt, dash pattern = on 0pt off 12 pt, line cap=round](avav) to [bend left =15] (avant);


\end{scope}

\begin{scope}[shift={(-6,0)}]%cerlce de gauche
	%Le cerlce et les pts
	\draw (0,0) circle (\r);
	\node (j) at ({15:\r+0.3}) {j};
	\node (sj) at ({-15:\r+0.4}) {$\sigma(j)$};
	\node (ssj) at ({-45:\r+0.4}) {$\sigma^2(j)$};
	\node (sqmuj) at ({45:\r+0.6}) {$\sigma^{q-1}(j)$};
	\node (sqmdj) at ({75:\r+0.6}) {$\sigma^{q-2}(j)$};
	\node (preced) at ({105:\r+0.6}) {};
	\node (prepre) at ({135:\r+0.6}) {};
	\node (suiv) at ({-75:\r+0.4}) {};
	\node (suisui) at ({-105:\r+0.6}) {};
	
	%Les flèches :
	\draw [line width = 0.75pt,->] (sqmdj) to [bend left = 20](sqmuj);
	\draw [line width = 0.75pt,->] (sqmuj) to [bend left = 20](j);
	\draw[line width = 0.75pt,->] (j) to [bend left = 20](sj) ;
	\draw [line width = 0.75pt,->] (ssj) to [bend left = 20] (suiv);
	\draw [line width = 0.75pt,->] (preced) to [bend left = 20] (sqmdj);	
	\draw [line width = 0.75pt,->] (sj) to [bend left = 20](ssj);
	
	\draw [line width = 3pt, dash pattern = on 0pt off 12 pt, line cap=round](prepre) to [bend left =15] (preced);
	\draw [line width = 3pt, dash pattern = on 0pt off 12 pt, line cap=round](suiv) to [bend left =15] (suisui);
\end{scope}

\begin{scope}[red]
%Les flèches du millieux
	\draw[line width = 0.75pt,->] (j) to[bend left = 15] node [midway,above]{$\sigma\circ\tau$} (si)  ;
	\draw[line width = 0.75pt,->] (i) to[bend left = 15] node [midway,below]{$\sigma\circ\tau$} (sj);
	
\begin{scope}[shift={(-6,0)}]%flèche de gauche rouge
	%Le cerlce et les pts
	\node (j1) at ({15:\r-0.3}) {};
	\node (sj1) at ({-15:\r-0.3}) {};
	\node (ssj1) at ({-45:\r-0.3}) {} ;
	\node (sqmuj1) at ({45:\r-0.3}){};
	\node (sqmdj1) at ({75:\r-0.3}){};
	\node (preced1) at ({105:\r-0.3}){};
	\node (prepre1) at ({135:\r-0.3}){};
	\node (suiv1) at ({-75:\r-0.3}){};
	\node (suisui1) at ({-105:\r-0.3}){};
	
	%Les flèches :
	\draw [line width = 0.75pt,->] (sqmdj1) to [bend right = 20](sqmuj1);
	\draw [line width = 0.75pt,->] (sqmuj1) to [bend right = 20](j1);
	\draw [line width = 0.75pt,->] (ssj1) to [bend right = 20] (suiv1);
	\draw [line width = 0.75pt,->] (preced1) to [bend right = 20] (sqmdj1);	
	\draw [line width = 0.75pt,->] (sj1) to [bend right = 20](ssj1);
	
	\draw [line width = 3pt, dash pattern = on 0pt off 12 pt, line cap=round](prepre1) to [bend right =15] (preced1);
	\draw [line width = 3pt, dash pattern = on 0pt off 12 pt, line cap=round](suiv1) to [bend right =15] (suisui1);
\end{scope}

\begin{scope}[shift={(6,0)}] % Cercle de droite

    % Définition des nœuds :
    \node (avav1) at ({315:\r-0.3}) {};
    \node (avant1) at ({285:\r-0.3}) {};
    \node (spmdi1) at ({255:\r -0.3}) {};
    \node (spmui1) at ({225:\r -0.3}) {};
    \node (i1) at ({195:\r -0.3}) {}; 
    \node (si1) at ({165:\r -0.3}) {};
    \node (ssi1) at ({135: \r -0.3}) {};
    \node (apres1) at ({105:\r-0.3}) {};
    \node (aprapr1) at ({75:\r-0.3}) {};

    % Les flèches :
    \draw[line width = 0.75pt,->] (spmdi1) to [bend right = 20] (spmui1);
    \draw[line width = 0.75pt,->] (spmui1) to [bend right = 20] (i1);
    \draw[line width = 0.75pt,->] (si1) to [bend right = 20] (ssi1);
    \draw[line width = 0.75pt,->] (ssi1) to [bend right = 20] (apres1);
	\draw[line width = 0.75pt,->] (avant1) to [bend right = 20] (spmdi1);    
    
    \draw [line width = 3pt, dash pattern = on 0pt off 12 pt, line cap=round](apres1) to [bend right =15] (aprapr1);
	\draw [line width = 3pt, dash pattern = on 0pt off 12 pt, line cap=round](avav1) to [bend right =15] (avant1);
\end{scope}



\end{scope}

\end{tikzpicture}

\end{flushleft}

    \end{minipage}
}
\end{center}
\end{itemize}
\end{preuve}


\section{Groupes Diédraux}

$\dsp\Omega=
\begin{pmatrix}
\cos\left( \frac{2\pi}{n}\right) & -\sin\left( \frac{2\pi}{n}\right)\\
\sin\left( \frac{2\pi}{n}\right) & \cos\left( \frac{2\pi}{n}\right)
\end{pmatrix}$  est la rotation d'angle $ \dfrac{2\pi}{n}$. On note aussi $S=\begin{pmatrix}
1&0\\
0&-1
\end{pmatrix}$ la symétrie d'axe $(O_{x})$

\begin{prop}
\begin{enumerate}
\item $\Omega ^{n}=e$ et $S^{2}=e$
\item $S\Omega S = \Omega^{-1}$, et donc $S\Omega^{-1} =\Omega S $
\end{enumerate}
\end{prop}

\begin{exemple}
\begin{enumerate}[label = $\bullet$]
\item $n=2$ : $D_{2}\simeq \znz \times \znz$
\item $n=3$ : \par
		On a $\Omega^{-1}S\Omega = S\Omega^{2}=Sr^{-1}=rS$ et \par 
		$r^{-2}sr^{2}=r^{-2}r^{-2}s=r^{2}s$\par
		Et donc $D_{3}\simeq \mathcal{S}_{3}$
\item $n=4$ :\par
\begin{center}
	\begin{tikzpicture}

\fill[gray!10] (-4,-4) rectangle (4,4);

% Les éléments de Dn en couleurs
\draw[thick,blue] (-2,-2) -- (2,2);   % Diagonale (angle 45°)
\draw[thick,red] (-2,2) -- (2,-2);    % Diagonale (angle -45°)
\draw[line width=3pt,green,opacity=0.5] (-2.42,0) -- (2.42,0);
\draw[line width=3pt,green,opacity=0.5] (0,-2.42) -- (0,2.42);

% Axes principaux
\draw[thick,->] (-3,0) -- (3,0) ;  % Axe x
\draw[thick,->] (0,-3) -- (0,3) ;  % Axe y

% Le carré
\draw[thick,black] (0,2)--(2,0);
\draw[thick,black] (0,2)--(-2,0);
\draw[thick,black] (-2,0)--(0,-2);
\draw[thick,black] (0,-2)--(2,0);

% Cercles représentant les éléments du groupe
\draw[thick,dashed] (0,0) circle (2); 

\node[red] at (-2.3,2.3) {$r^3s$};
\node[blue] at (2.3,2.3) {$rs$};
\node[left ,green] at (0,2.3) {$r^2s$};
\node[below, green] at (-2.3,0) {$s$};

\end{tikzpicture}
	\end{center}
	\end{enumerate}
	\end{exemple}

\begin{theoreme}
Soit $n\neq 2$, $D_{n}=<s,r>$ alors :\par
\begin{center}
$D_n=\{ e,r,r^2,\cdots,r^{n-1},s,rs,\cdots,r^{n-1}s \}$
\end{center}
En particulier, $|D_{n}|=2n$ et $<r>$ est distingué dans $D_{n}$
\end{theoreme}

\begin{preuve}
Les éléments $e,r,r^2,\cdots,r^{n-1}$ sont distincts deux à deux, de même que le sont $s,sr,sr^2,\cdots,sr^{n-1}$. Il reste alors qu'a remarquer que ces deux ensembles sont disjoints, par exemple au moyen d'un déterminant de matrice.
\end{preuve}

Soit $g\in < r,s >$ :  c'est un mot en $r,s,r^{-1},$, en utilisant $sr=r^{-1}s$ on se ramène à un mot de "réduit" de la forme $e,r,r^2,\cdots,r^{n-1}$ ou $s,sr,sr^2,\cdots,sr^{n-1}$.

\remarque{$D_{n}$ est aussi engendré par $r$ et $rs$, en effet $r=rs\cdots$}

\begin{exo}
Soit G un groupe engendré par deux éléments a et b qui vérifient, $o(a)=n$, $o(b)=2$ et $o(ab)=2$ alors G est isomorphe à $D_{n}$
\end{exo}

\begin{preuve}
$ab=ba^{-1} \Rightarrow b\notin <a>$ et $G=\{e,a,a^{2},\cdots,a^{n-1},b,ba,\cdots,a^{n-1}b \}$ Par exemple J calais (Chapitre groupes diédraux)
\end{preuve}

\remarque{En TD on identifiera la liste des sous groupes normaux de $D_{n}$}

\begin{exo}
\begin{itemize}[label=$\bullet$]
\item Soit $n\neq3$, alors $ Z(\dn)= \{ e,r^{n/2} \} $ si n est pair et $e$ sinon.
\item $D(D_{1}) = \{ e \}$ et $D(D_{2}) = \{ e \}$
\item $\forall n\neq 3, ~D(\dn) = \langle r^{2}\rangle$
\end{itemize}
\end{exo}

\begin{preuve}
\begin{itemize}
\item $[r^i,r^j]=e$
\item $[r^{i},r^{j}s]=r^{i}r^{j}sr^{-i}(r^{j}s)^{-1} = r^{i+j}sr^{-i}sr^{-j}=r^{2i}$
\item $[r^{i}s,r^{j}s]=r^{i-j}$
On a bien $D(\dn) = \langle r^{2}\rangle$
\end{itemize}
\end{preuve}

\section{Classification des groupe d'ordre 8}

\subsection{Définition de $Q_{8}$}

$Q_{8}= \langle I,J \rangle$ où $I = \begin{pmatrix}
i&0\\
0&-i
\end{pmatrix}$, et $I = \begin{pmatrix}
0&1\\
-1&0
\end{pmatrix}$

\begin{prop}
\begin{enumerate}
\item $o(I)=4$, $J^2=-I(\Leftarrow o(J)=4$
\item $JI=I^{-1}J$
\end{enumerate}
\end{prop}


\begin{prop}
$Q_{8}= \{ e,I,I^{2},I^{3},J,IJ,I^{2}J,I^{3}J \}$
\end{prop}
\begin{preuve}
Similaire à celle faite pour $\dn$
\end{preuve}


\begin{theoreme}
Soit G un groupe non abélien d'ordre 8.\par
Si G possède un seul élément d'ordre 2 alors : $G\simeq Q_{8}$, sinon, $G\simeq D_{4}$
\end{theoreme}


\begin{preuve}
Tout les éléments de G ne peuvent être abéliens à la fois, car G est non abélien. Dès lors, il existe un élément i d'ordre 4 (8 étant exclu car $G\neq \Z/8\Z$ abélien). On note $H:=\langle i \rangle$\par 
Soit $j\in G~,~ j\notin H$, on a $G = \{ 1,i,i^2,i^3 \} \cup \{ j,ji,ji^2,ji^3 \} = H \cup jH$.\par
Comme $H\triangleleft G$ vu $[G:H]=2$, on a $jij^{-1} \in H$

\begin{enumerate}
\item Si G possède un seul élément d'ordre 2, c'est  $i^{2}\in \langle i \rangle$. On a de plus $o(j)=o(ij)=o(i^{2}j)=o(ij^{3})=4$, et $ji=i^{-1}j$, on vérifie alors que $G\simeq Q_{8}$
\item
Si G possède au moins 2 éléments d'ordre 2 alors il existe dans $ \{ j,ji,ji^2,ji^3 \}$ un élément d'ordre 2, notons le $j_{0}$, (par exemple$i^{2}j$ fonctionne)
Comme précédemment 

\end{enumerate}
\end{preuve}
\section{Action de groupes}
\ju
\begin{definition}[Action] Soit $G$ un groupe et $X$ un ensemble non vide. Une opération de $G$ sur un ensemble $X$ est une application $G\times X\longrightarrow  X,\ (g,x)\mapsto  g\cdot x$
qui vérifie
\begin{itemize}
\item $g_1\cdot(g_2\cdot x)=(g_1g_2)\cdot x,\ \forall g_1,g_2\in G,\ \forall x\in G$
\item $e\cdot x=x,\ \forall x\in X$
\end{itemize}
\end{definition}
\ju 
\remarque{On a défini ici une action de $G$ à gauche. On peut définir une action de $G$ à droite en demandant cette fois ci : $(x\cdot g_1)\cdot g_2=x\cdot (g_1g_2)$}
\jus
\begin{prop}
\begin{itemize}
\item $\forall g\in G$, l'application $\gamma_g:X\longrightarrow  X,\ x\mapsto  g\cdot x$ est une bijection (d'inverse $\gamma_{g^{-1}}$)
\item L'application $G\longrightarrow \text{Bij}(X),\ x\mapsto  \gamma_x$ est un morphisme de groupes. Réciproquement, tout morphisme de groupes $\lambda:G\longrightarrow  \text{Bij}(X)$ définit une action de $G$ sur $X$ en posant $g\cdot x=(\lambda(g))(x)$.
\end{itemize}
\end{prop}
\jus
\remarque{On étudira le cas particulier où $X=G$, il s'agit d'un cas très intéressant. On peut avoir $G\longrightarrow \text{Bij}(G)$ et même des exercices où $G\longrightarrow \text{Aut}(G)$.}
\jus 
\begin{exemple}
\begin{itemize}
\item[1)] $G$ opère sur $G$ par translation à gauche $G\times G,\ (g,x)\longrightarrow  g\cdot x=gx$. 
\item[2)] $G$ opère sur $G$ par conjugaison  $G\times G\longrightarrow  G,\ (g,x)\mapsto  g\cdot x=gxg^{-1}$. Ici, l'application $G\longrightarrow  G,\ x\mapsto  gxg^{-1}$ est un automorphisme de $G$. Donc on a ici $G\longrightarrow  \text{Aut}(G)$. 
\end{itemize}
\end{exemple}
\jus 
\begin{definition}[automorphisme intérieur] 
L'application $i_g:G\longrightarrow  G,\ x\mapsto  gxg^{-1}$ s'appelle l'automorphisme intérieur associé à $g$. 
\end{definition}
\jus 
Exercice. L'ensemble $\text{Int}(G)$ des automorphismes intérieurs de $G$ forme un sous-groupe de $\text{Aut})(G)$. 
\jus 
Lemme. $\text{Int}(G)\simeq G/Z(G)$
\ju 
\begin{preuve}
On considère le morphisme 
\fonctions{\varphi}{G}{\text{Int}(G)}{g}{i_g}
\begin{itemize}
\item L'application $\varphi$ est évidemment surjective. 
\item 
\begin{align*}
g\in\ker(\varphi) &\ssi i_g=\text{Id}_G
\\ &\ssi \forall x\in G,\ i_g(x)=gxg^{-1}=x
\\ &\ssi \forall x\in G,\ gx=xg
\\ &\ssi g\in Z(G).
\end{align*}
On conclut en appliquant le 1er théorème d'isomorphisme.
\end{itemize}
\end{preuve}
\begin{exemple} \textbf{(Suite des exemples)}
\begin{itemize}
\item[3)] Soit $H<G$ (pas forcément distingué). Soit l'application 
\fonctions{f}{G\times (G/H)_{\text{gauche}}}{(G/H)_{\text{gauche}}}{(g,xH)}{(gx)H}
Cette application est bien définie. En effet, si $gH=xH$ alors $y=xh$ où $k\in H$, et ensuite $g(yH)=g(xhH)=g(xH)=(gx)H$. 
\item[4)] $G:=\text{SL}_2(\R)$ agit sur $\mathbb{H}:=\{z\in\C,\ \text{Im}(z)>0\}$ via 
\fonctions{f}{G\times \mathbb{H}}{\mathbb{H}}{\G(\begin{pmatrix}a&b\\ c&d\end{pmatrix} ,z\D)}{\frac{az+b}{cz+d}}
\item[5)] $O_n(\R):=\{M\in M_n(\R),\ M^\top M=I_n\}$ agit sur $\mathbb{S}^n:=\{(x_1,\cdots,x_n)\in\R^n,\ x_1^2+\cdots+x_n^2=1\}$ via 
\fonctions{g}{\text{GL}_n(\R)\times\mathbb{S}^n}{\mathbb{S}^n}{(M,x)}{Mx}
\item[6)] $D_n$ (groupe diédral) agit sur l'ensemble des sommets du polygôle régulier à $n$ côtés. 
\item[7)] $\text{GL}_n(\R)$ agit sur l'ensemble des matrices symétriques $S_n(\R)$ via 
\fonctions{h}{\text{GL}_n(\R)\times S_n(\R)}{\mathbb{S}^n}{(g,x)}{g^{\top}xg}
\end{itemize} 
\end{exemple}
\jus 
\subsection{Stabilisateur, orbite}
\jus 
\begin{definition}[stabilisateur] 
Soit $x\in X$. Le stabilisateur de $x$ dans $G$ est 
$$\text{Stab}_G(x):=\{g\in G,\ g\cdot x=x\}.$$
\end{definition}
\ju 
Exercice. $\text{Stab}_G(x)<G$. 
\jus 
Maintenant, introduisons la relation sur $X$ suivante : 
$$x\mc{R}y\ssi\exists g\in G,\ y=g\cdot x.$$
\ju 
Exercice. $\mc{R}$ est une relation d'équivalence. 
\ju 
\begin{definition}[] 
Soit $x\in X$. $\text{Orb}_G(x)$ est la classe d'équivalence de $x$ par la relation $\mc{R}$. Autrement dit : 
$$\text{Orb}_G(x)=\{g\cdot x,\ g\in G\}$$
\end{definition}
\ju 
Retour sur les 7 exemples. 
\ju 
\begin{itemize}
\item[1)] $\text{Stab}_G(x)=\{x\}n\ \text{Orb}_G(x)=G$. 
\item[2)] $\text{Stab}_G(x)=\{g\in G,\ gxg^{-1}=x\}=\{g\in G,\ gx=xg\}$, appelé le "centraliseur" de $x$, noté $C_G(x)$. 
$$\text{Orb}_G(x):=\{gxg^{-1},\ g\in G\}$$
est la "classe de conjugaison" de $x$.
\item[3)] \fonctions{f}{G\times (G/H)_{\text{gauche}}}{(G/H)_{\text{gauche}}}{(g,xH)}{(gx)H}
$$\text{Stab}_G(xH)=xHx^{-1}$$
$$\text{Orb}_G(xH):=(G/H)_{\text{gauche}}$$
\item[4)] \fonctions{f}{\text{SL}_2(\R)\times \mathbb{H}}{\mathbb{H}}{\G(\begin{pmatrix}a&b\\ c&d\end{pmatrix} ,z\D)}{\frac{az+b}{cz+d}}
$$\text{Stab}_G(i)=\G\{\begin{pmatrix}a&-b\\ b&a\end{pmatrix},\ a^2+b^2=1\D\}\simeq \text{SO}_n(\R)=\G\{\begin{pmatrix}\cos(\theta)&-\sin(\theta)\\ \sin(\theta)&\cos(\theta)\end{pmatrix},\ \theta\in\R\D\}$$
$$\text{Orb}_G(i):=\mathbb{H}.$$
\item[6)]
\item[7)] \fonctions{h}{\text{GL}_n(\R)\times S_n(\R)}{\mathbb{S}^n}{(g,M)}{g^{\top}Mg}
$$\text{Stab}_G(M)=\{\text{groupes des isométries par la forme quadratique induite par }M\}$$
$$\text{Orb}_G(xH):=\{M^{\top}\in S_n(\R),\ \text{signature}(M)=\text{signature}(M')\}$$
\end{itemize} 
\jus 
Fait important : Si $y\in\text{Orb}_G(x)$ alors on peut relier $\text{Stab}_G(x)$ et $\text{Orb}_G(y)$. On a 
$$\text{Stab}_G(x)=g\text{Stab}_G(x)g^{-1}.$$
Autre propriété importante :
\ju 
\begin{theoreme}
Soit $x\in X$. L’application 
\fonctions{V}{(G/\text{Stab}_G(x))_{\text{gauche}}}{\text{Orb}_G(x)}{g\text{Stab}_G(x)}{g\cdot x}
est bien définie et c’est une bijection. Attention, $V$ n’est pas un morphisme de groupes. 
\end{theoreme}
\jus 
\begin{preuve}
Soit $S:=\text{Stab}_G(x)$. Soit $(g,g')\in G^2$ tel que $g'S=Sg$. Alors $g'=gS$ où $s\in S$. Ainsi, $g'\cdot x=(gS)\cdot x=g\cdot (x\cdot s)=g\cdot x$.
\ju 
$V$ est surjective par définition. 
\ju 
$V$ est injective : : Soit $(g,g')\in G^2$ tel que $g\cdot x=g'\cdot x$. On a $(g')^{-1}\cdot (g\cdot x)=x$. Or $(g')^{-1}\cdot (g\cdot x)=((g')^{-1}g)\cdot x$, donc $(g')^{-1}g\in S$, donc $g\in g'S$, donc $gS=g'S$.  
\end{preuve}
\ju 
\begin{theoreme}
\begin{itemize}
\item[1)] $\forall x\in X,\ \G|\text{Orb}_G(x)\D|=\frac{|G|}{|\text{Stab}_G(x)|}=[G:\text{Stab}_G(x)]$. 
\item[2)] Si $X$ est fini et si $\{x_1,\cdots,x_r\}$ est un ensemble de représentants des orbites par la relation $\mc{R}$, alors 
$$|X|=\sum_{i=1}^{r}|\text{Orb}_G(x_i)|= \sum_{i=1}^{r}\frac{|G|}{|\text{Stab}_G(x_i)|}.$$
\end{itemize}
\end{theoreme}

\end{document}
